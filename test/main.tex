%% <head>
\documentclass{article}
\usepackage{amsmath,amssymb,multicol,ulem,makecell,commath,pifont}
\usepackage[a4paper,margin=0.1in]{geometry}
\usepackage[inline]{enumitem}

% math-mode version of "c" column type
\newcolumntype{C}{>{$}c<{$}}

\newcommand{\Z}{\mathbb{Z}}
\newcommand{\R}{\mathbb{R}}

% draw the vertical divider
\setlength{\columnseprule}{0.2pt}

% remove all section numbers without having to add *
\setcounter{secnumdepth}{0}

% flush-left all itemize and enumerate lists
\setlist[itemize,enumerate]{leftmargin=*}
\setlength{\parindent}{0pt}

% <headings>
\makeatletter
% ensure non-recursive remaps
\let\TMPsection\section
\let\TMPsubsection\subsection
\renewcommand{\TMPsection}{\@startsection{section}{2}{0pt}{1ex}{1.2ex}{\normalsize\bfseries\fbox}}
\renewcommand{\TMPsubsection}{\@startsection{subsection}{3}{0pt}{1ex}{0.8ex}{\footnotesize\bfseries\uline}}
% h1-h4 declarations
\renewcommand{\title}{\@startsection{section}{1}{0pt}{1ex}{1.2ex}{\Large\bfseries\uline}}
\renewcommand{\section}[1]{\TMPsection{#1}}
\renewcommand{\subsection}[1]{\TMPsubsection{#1}}
\renewcommand{\paragraph}{\@startsection{paragraph}{4}{0pt}{1.5ex}{-0.8em}{\bfseries}}
% </headings>

% force document-wide displaystyle by default
\everymath{\displaystyle}

\newcommand{\documentinit} {
	\footnotesize
	\setlength{\abovedisplayskip}{2pt}
	\setlength{\belowdisplayskip}{2pt}
	\setlength{\abovedisplayshortskip}{0pt}
	\setlength{\belowdisplayshortskip}{0pt}
}

% global shorthands
\newcommand{\mat}[1]{\begin{pmatrix}#1\end{pmatrix}}
\newcommand{\pmat}[1]{\begin{matrix}#1\end{matrix}}
\let\TMPepsilon\epsilon\def\epsilon{\varepsilon}
\newcommand{\build}{\ding{115}\ }
\let\div\relax
\let\curl\relax
\DeclareMathOperator{\div}{div}
\DeclareMathOperator{\curl}{curl}

% multivariable calculus!
\renewcommand{\d}{\,\mathrm{d}}
\newcommand{\p}{\partial}
\newcommand{\g}{\nabla}
\newcommand{\dd}[2]{\frac{\d{#1}}{\d{#2}}}
\newcommand{\pp}[2]{\frac{\p{#1}}{\p{#2}}}

% </head>

\begin{document}
\documentinit

\begin{multicols*}{2}

	\title{MA2104 Multivariable Calculus}

	%% set a {\mathbf{a}} set b {\mathbf{b}} set c {\mathbf{c}} set dot {\cdot}
	%% set proj {\mathrm{proj}} set comp {\mathrm{comp}}
	\section{Fundamentals}
	$\proj_\b\a = \frac{\a\dot\b}{\b\dot\b}\b$
	\quad;\quad
	$\comp_\b\a = \frac{\a\dot\b}{||\b||}\b$ (length of projection) \\\\
	\build let $k:=$ len. of proj. of $\a$ onto $\b$. By
	Pythagoras, $k=\frac{\a\dot\b}{\b\dot\b}$

	Area of parallelepiped $= |\a\dot(\b\times\c)|$ (order doesn't matter)
	%% unset a b dot proj comp


	\section{Integration hacks}
	Trigonometric reduction.
	\begin{itemize}
		\item $\int\sin^nx\d{x}
			      = \frac{n-1}n\int\sin^{n-2}x\d{x}
			      - \frac{\cos x\sin^{n-1}x}n$
		      % source: ../python/shorthands.py
		\item $\int\cos^nx\d{x}
			      = \frac{n-1}n\int\cos^{n-2}x\d{x}
			      + \frac{\sin x\cos^{n-1}x}n$
		      % source: ../python/shorthands.py
		\item $\int\csc^nx\d{x}
			      = \frac{n-2}{n-1}\int\csc^{n-2}x\d{x}
			      - \frac{\csc^{n-2}x\cot x}{n-1}$
		      % source: https://www.integral-calculator.com
		\item $\int\sec^nx\d{x}
			      = \frac{n-2}{n-1}\int\sec^{n-2}x\d{x}
			      + \frac{\sec^{n-2}x\tan x}{n-1}$
		      % source: https://www.integral-calculator.com
		\item $\int\tan^nx\d{x}
			      = \frac{\tan^{n-1}x}{n-1}
			      - \int\tan^{n-2}x\d{x}$
		      % source: ../python/shorthands.py
		\item $\int\cot^nx\d{x}
			      = -\frac{\cot^{n-1}x}{n-1}
			      - \int\cot^{n-2}x\d{x}$
		      % source: ../python/shorthands.py
	\end{itemize}


	\section{Parameterizing Lines $R(t)$}
	Let curve $C:R(t)=(x(t),y(t))$. Tangent vector is
	$R'(t):=(x'(t),y'(t))$ \\\\
	\textbf{Velocity}
	$\textstyle\norm{R'(t)}=\sqrt{x'(t)^2+y'(t)^2}$
	\quad;\quad
	\textbf{Arc Length} $\int_{t_1}^{t_2}\norm{R'(t)}\d t$
	\paragraph{Unit Speed Parameterization} is an $R(t)$ such that
	$\frac{\d{\text{(ArcLen)}}}{\d t} = 1$.

	\section{Level Curves}
	$k$-level set of $f(x,y)$ is the set $\{(x,y)\in\R^2:f(x,y)=k\}$

	\section{Multivariable Limits}
	%% set Lim #1#2#3#4{\lim_{(#1,#2)\to(#3,#4)}}
	Let $f:\R^2\to\R$. $\Lim xyab f(x,y)=L$ if for any
	$\epsilon>0$, $\exists\;\delta > 0$ such that \\\\
	$0<\norm{\mat{x\\y}-\mat{a\\b}}<\delta \implies |f(x,y)-L|<\epsilon$
	\subsection{Showing that limits do not exist}
	Pick functions that converge to the same limits for each variable,
	and show that the limit reached changes depending on the function
	chosen.
	$$\Lim xy00 \frac{xy}{x^2+y^2};\ \text{send }x\to y\
		(f(x,y)\to\tfrac1 2);\ \text{send }x\to2y\ (f(x,y)\to\tfrac2 3)$$
	%% unset Lim

	\subsection{Showing that limits exist}
	\paragraph{Squeeze theorem} If $(g(x)\leq f(x)\leq h(x)\ \forall x)
		\land (\lim_Pg(x) = \lim_Ph(x) = L)$, then $\lim_Pf(x) = L$ too. ($P$ can
	be $x\to\infty,\ x\to 1$, etc.)

	\section{Continuity}
	%% set Lim #1#2#3#4{\lim_{(#1,#2)\to(#3,#4)}}
	Let $f:\R^2\to\R$. $f$ is continuous at $(a,b)$ if $\Lim xyabf(x,y)=f(a,b)$
	\paragraph{Theorem 2.4} if $f$ and $g$ are both cont. at
	$(a,b)$, then $h = f\circ g$ is cont. too.

	%% unset Lim

	\section{Partial Derivative $\p$}
	\paragraph{Def. 2.6} $\frac{\p f}{\p x}=f_x(x,y):=\frac{f(x+h,y)-f(x,y)}h$
	\paragraph{Clairaut's Thm} If $f$ is defined on $D$, and
	$f_{xy}, f_{yx}$ are both cont, $f_{xy}=f_{yx}$.

	\section{Differentiability}
	\paragraph{Def. 2.12} $f$ is diff'able at $(a,b)$ if
	$\exists$
	a linear map $L:\R^2\to\R$ s.t.
	%% set Lim #1#2#3#4{\lim_{(#1,#2)\to(#3,#4)}}
	$$\Lim hk00 \frac{f(a+h,b+k)-f(a,b)-L(h,k)}{\sqrt{h^2+k^2}}=0$$
	%% unset Lim
	$L$ is the total derivative of $f$ at $(a,b)$. \\\\
	\build as $(h,k)$ shrinks, we get closer to $(a,b)$, and
	$f(a+h,b+k)-f(a,b)$ behaves increasingly linearly (like $L(h,k)$).
	In 1-D, the differrentiable curve approaches a straight line near
	to the point of differentiation.
	\\\\
	As $L$ depends on $f$ and point $(a,b)$, we sometimes write $L$ as
	$Df_{(a,b)}$. wgt.
	\paragraph{Def. 2.13} $\{(x,y,L(x-a,y-b)+f(a,b))\}$ is tangent plane
	to $\{(x,y,f(x,y))\}$ at the point $(a,b)$.
	\paragraph{Prop. 2.14} If $f$ is diff'able, then
	$L(h,k)=f_x(a,b)h+f_y(a,b)k$

	\paragraph{Linear Approximation} $f(a+h,b+k)\approx f(a,b)+L(h,k)$
	\section{Chain Rule}
	\paragraph{One-variable CR} $\dd yt=\dd yx\dd xt$
	\paragraph{Two-variable CR$_{(t)}$} $\dd zt=\pp zx\dd xt+\pp zy\dd yt$
	\paragraph{Two-variable CR$_{(s,t)}$} $\pp zs=\pp zx\pp xs+\pp zy\pp ys$
	\quad;\quad $\pp zt=\pp zx\pp xt+\pp zy\pp yt$ \\\\
	\build use $\d x$ only when it gives full story (that's
	probably why $\p$ is partial).

	\section{Directional derivative $Df_{(a,b)}$}
	Literally $L$. Also, since $L$ takes a 2-D input, it can technically
	take a 2-D vector just as well:
	%% set u {\mathbf{u}}
	$$Df_{(a,b)}(\u)=\lim_{h\to0}\frac{f(a+hu_1,b+hu_2)-f(a,b)}{h}\quad\text{where \textbf{u}(is a)\textbf{nit vector}}$$
	\build this can be sherlocked from the mentioned differentiability condition.

	\section{Gradient vector $\g f$}
	\subsection{Definition}
	Again, literally $L$. Since $L:\R^2\to\R$, we can just as well
	define $\g f$ as:
	$\g f(a,b):=\mat{f_x(a,b)\\f_y(a,b)}$, and so if $\u=\mat{h\\k}$,
	$$L(h,k)=Df_{(a,b)}(\u)=\g
		f(a,b)\cdot\u=f_x(a,b)h+f_y(a,b)k$$

	\subsection{Perpendicular vector to level sets}
	If $\g f(a,b)\neq0$, then $\g f(a,b)\perp$ level curve of $f$ that
	contains $(a,b)$
	\paragraph{Result from notes:}
	$$\frac\d{\d t}f(x(t),y(t))=\g f(x(t),y(t))\cdot\mat{x'(t)\\y'(t)}\quad\text{(Chain Rule)}$$
	(3-D) if $\g f(a,b,c)\neq0$, then $\g f(a,b,c)\perp$ level set of
	$f$ that contains $(a,b,c)$.
	%% set r {\mathbf{r}} set a {\mathbf{a}} set n {\mathbf{n}}
	Hence, the tangent plane's equation at $(a,b,c)$ with position
	vector $\a$:

	$$\g f(a,b,c)\cdot(\r-\a) = 0\quad\text{ (recall }\n\cdot(\r-\a)=0)$$
	%% unset r n a

	\section{Implicit partial differentiation}
	Not much, so here's an example of taking $\pp zx$:
	$$x^3+y^3+z^3+6xyz=1\;\longrightarrow\;3x^2+0+3z^2\pp zx+6\left(yz+xy\pp zx\right)=0$$
	very useful when you don't actually have to evaluate any one variable.

	\section{Critical Point}
	All directional derivatives of $f$ at $(a,b)$ are 0 iff $f_x(a,b)=f_y(a,b)=0$.
	Such $(a,b)$ are called critical points.

	\paragraph{Extreme Value Theorem} if $f:D\to\R$ is cont. on a closed
	and bounded set $D\subseteq\R^2$, then $f$ has at least one global
	maximum, and one global minimum.

	\subsection{Computing global maxima}
	\textbf{1.} Get all crit. pts. of $f$ and their values \textbf{2.} Find extreme
	values of $f$ along the boundary of $D$. \textbf{3.} Select the
	largest value.


	\section{Line integrals}
	Let curve $C$ be parameterized by $R(t)=(x(t),y(t)),\ a\leq t\leq b$.
	$$\int_Cf(x,y)\d s:=\int_a^bf(x(t),y(t))\norm{R'(t)}\d t$$
	\paragraph{Observation}
	$\textstyle\int_C1\d s=\int_a^b\norm{R'(t)}\d t=\text{Arc length of
			$C$ by definition}$

	\section{Vector fields}
	%% set C {\mathbf{C}} set F {\mathbf{F}} set r {\mathbf{r}} set n {\mathbf{n}}
	Line integrals over a vector field computes work done. Let $\C$ be
	an oriented curve that traces $R(t)$ as $t$ goes from $a$ to $b$ in $a\leq t\leq b$.
	$$W=\int_\C\F\cdot\d\r:=\int_a^b\F(x(t),y(t),z(t))\cdot R'(t)\d t=\int_a^bX\d x+Y\d y+Z\d z$$

	\subsection{Conservative vector fields}
	$\F$ is conservative on an open domain $D$ if it can be written as $\g f$.

	\paragraph{Gradient Theorem} $\int_\C\g f\cdot\d\r=f(R(b))-f(R(a)),\
		\C:R(t),\ t\in[a,b]$
	Hence it's worth trying to write $\F=\g f$, to get
	$\int_\C\F\cdot\d\r=f(R(b))-f(R(a))$.

	\subsection{Testing for conservativity}
	\paragraph{\texttt{!=}} Find $\C_1,\C_2$ s.t.
	$\textstyle\int_{\C_1}\F\cdot\d\r\neq\int_{\C_2}\F\cdot\d\r$, or
	a loop $\C$ s.t. $\textstyle\int_{\C}\F\cdot\d\r\neq0$
	\paragraph{\texttt{==}} Find a $\g f$ such that $\F=\g f$.
	\paragraph{Theorem} Let $\F\in D$  where $D$ is open and simply
	connected. Then $\pp Xy=\pp Yx\land\pp Xz=\pp Zx\land\pp Yz=\pp Zy\iff\F$ is conservative.
	\section{Green's Theorem}
	Let $D$ be a region bounded by a counter-clockwise curve $\C$. Then
	$$\int_\C\F\cdot\d\r=\iint_D\left(\pp Yx-\pp Xy\right)\d A$$
	This can be used backwards to reduce $\textstyle\iint_D1\d A$ to
	$\textstyle\int_\C\F\cdot\d\r$.
	\paragraph{Convenience routines}
	$$A=\int_\C x\d y=\int_\C -y\d x=\frac12\left(\int_\C x\d y+y\d x\right)$$
	This is a direct result of Green's Theorem by picking nice values of $\F$.

	\paragraph{Divergence} $\div\F:=\pp Xx+\pp Yy\left(+\pp Zz\right)_{\text{(for 3-D)}}$

	\paragraph{Green's Theorem V2} (total outward flux)
	$$\int_\C\F\cdot\n\d s=\iint_D\div\F\d A,\quad\text{where
			$\norm\n=1$ and $\n\perp\C$ outwards}$$
	%% set comp {\mathrm{comp}}
	Note that $\F\cdot\n=\comp_\n\F$
	%% unset comp

	\section{Surface integrals}
	\paragraph{Parameterizing surfaces} $R(u,v)=(x(u,v),y(u,v),z(u,v))$
	$$R_u(u,v):=\mat{x_u(u,v)\\y_u(u,v)\\z_u(u,v)}\text{, and }\;
		R_u(u,v):=\mat{x_u(u,v)\\y_u(u,v)\\z_u(u,v)}$$
	\paragraph{Defining surface integrals} (S is parameterized by $R(u,v)$)
	$$\iint_Sf(x,y,z)\d S:=\iint_Df(x(u,v),y(u,v),z(u,v))\norm{R_u\times R_v}\d A$$
	\paragraph{Corollary} if $z=g(x,y)$,
	$=\iint_Df(x,y,g(x,y))\sqrt{\textstyle\left(\pp
			gx\right)^2+\left(\pp gy\right)^2+1}\d A$

	%% set S {\mathbf{S}}
	\paragraph{Surface area} $A(S)=\iint_S1\d S$
	\paragraph{Flux of $\F$ across $\S$} $\iint_\S\F\cdot\d\S:=\iint_S\F\cdot\n\d S=\iint_D\F\cdot(R_u\times R_v)\d A$
	\paragraph{Corollary} if $z=g(x,y)$, $=\iint_D-X\pp gx-Y\pp gy+Z\d A$
	\section{Gauss' Theorem}
	$$\iint_\S\F\cdot\d\S=\iiint_E\div\F\d V$$ \\
	Flux through a surface = divergence of the solid it surrounds.
	\section{Stokes' Theorem}

	%% set i {\mathbf{i}} set j {\mathbf{j}} set k {\mathbf{k}}
	$\curl\F:=\textstyle
		\left(\pp Zy-\pp Yz\right)\i+
		\left(\pp Xz-\pp Zx\right)\j+
		\left(\pp Yx-\pp Xy\right)\k=
		\abs{\pmat{\i&\j&\k\\ \tfrac\p{\p x}&\tfrac\p{\p y}&\tfrac\p{\p
					z}\\X&Y&Z}}$
  \paragraph{Stokes' Theorem}
  $$\iint_\S\curl\F\cdot\d\S=\int_\C\F\cdot\d\r$$

	%% unset F C r n

	%% unset u
\end{multicols*}
\end{document}
